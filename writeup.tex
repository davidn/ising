\documentclass[12pt,a4paper,english]{article}
\usepackage[utf8x]{inputenc}
\usepackage{cite}
\usepackage{graphicx}
\usepackage{ucs}
\usepackage{babel}
\usepackage{fancyref}
\usepackage{relsize}
\usepackage{listings}
\usepackage{color}
\usepackage{mathpazo}
\usepackage[
  unicode=true,
  pdftitle={Redesigning Part IA Practicals},
  pdfauthor={David Newgas},
  bookmarks=true,
  bookmarksnumbered=false,
  bookmarksopen=false,
  breaklinks=false,
  pdfborder={0 0 0},
  backref=false,
  colorlinks=false
]{hyperref}

\author{David Newgas}
\title{Monte-Carlo Ising Model Simulation of a Two Dimensional Ferromagnet}
\begin{document}
\maketitle

\begin{abstract}
some text
\end{abstract}

\section{Introduction}
\label{sec:introduction}
In materials the interplay between minimising energy and maximising entropy leads to complex behaviour, including phase transitions where a physical property of a system changes abruptly as thermodynamic variable changes.  Thermodynamic analysis of such problems is only tractable for very simply problems, to which real life situations can be approximated.

The Ising model assumes an infinite lattice of points with ``spin'' pointing either up or down (taken as positive or negative for this paper).  The energy $E$ of the lattice is defined by:
\begin{equation}
\label{eqn:ising-energy}
E = - \sum_{\mathrm{neighbours}\: i,j} J \sigma_i \sigma_j - \sum_i \mu H \sigma_i
\end{equation}
Where $\sigma_i$ is the spin of the ith lattice point, $J$ is an interaction energy and $\mu H$ is a self-energy.  In the case where $\mu$ is a magnetic moment, $H$ is an external magnetic field and $J$ is a coupling constant the Ising model represents a ferromagnet where each domain interacts only with its nearest neighbours.

The Ising model can be analytically solved in one and two dimensions.  In both when $J>0$ a low temperature phase with spins aligning is seen with a transition to an unordered high temperature phase.  For the case where the model represents a ferromagnet, this low temperature phase is magnetic.  In the case where $J<0$ there is still (anti-aligned order) at low temperature and no order at high temperature.  As both are not magnetic there is no physical property in which to see a phase change.

Onsager \cite{onsager44} showed that in the two dimensional case $\sinh^2 \left(2J/kT_c\right)=1$, i.e.
\begin{equation}
\label{eqn:T-c}
T_c= \frac{2} {\ln \left( 1 + \sqrt{2}\right)}
\end{equation}

As no analytic solution exists in three dimensions, numerical methods for investigating the Ising model are of interest.  One such method is to directly simulate the time evolution of a finite lattice until equilibrium is reached.  The simulation requires randomness to ensure that the particular configuration reached is 

\section{Method}
\label{sec:method}

\subsection{Implementation Details}
\label{sec:implementation-details}

\section{Results}
\label{sec:results}

\section{Discussion}
\label{sec:discussion}

\section{Conclusions}
\label{sec:conclusions}

\bibliography{../../refs.bib}
\bibliographystyle{plain}

\appendix

\section{Source Code}

Following are files of code written for this project. Required also is the source from the gnuplot iostream interface \cite{gnuplot-iostream}.  The project is designed for use with a build infrastructure part of the submitted package, it can be build minimally with:
\lstset{
language=sh,
basicstyle=\small\tt,
tabsize=4,
title=\lstname,
breaklines=true,
frame=single,
breakatwhitespace=false,
keywordstyle=\color[rgb]{0,0,1},
commentstyle=\color[rgb]{0.133,0.545,0.133},
stringstyle=\color[rgb]{0.627,0.126,0.941},
xleftmargin=-5.75em,
xrightmargin=-5.75em
}
\begin{lstlisting}
g++ -O2 -o ising ising.cc lattice.cc ../gnuplot-iostream/gnuplot-iostream.cc -DOUTPUT_GNUPLOT -DDEFAULT_J=1 -DDEFAULT_muH=0 -DDEFAULT_kT=1 -DDEFAULT_SIZE=100 -DPACKAGE=\"ising\" -lboost_iostreams -lutil
g++ -O2 -o runs runs.cc ../gnuplot-iostream/gnuplot-iostream.cc -DOUTPUT_GNUPLOT -DDEFAULT_J=1 -DDEFAULT_muH=0 -DDEFAULT_kT=1 -DDEFAULT_SIZE=100 -DPACKAGE=\"ising\" -lboost_iostreams -lutil
\end{lstlisting}

\lstset{language=C++}
\lstinputlisting{src/ising.cc}
\lstinputlisting{src/lattice.h}
\lstinputlisting{src/lattice.cc}
\lstinputlisting{src/runs.cc}
\end{document}